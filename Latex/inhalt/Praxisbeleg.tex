% inhalt/Praxisbeleg.tex

\chapter{Einleitung und Problemstellung}
In der heutigen Zeit ist die effiziente Verwaltung von Kundendaten ein kritischer Erfolgsfaktor für Unternehmen. Aktuell erfolgt die Datenspeicherung im Betrieb auf einem \striche{Network Attached Storage} (NAS) in Form einer unstrukturierten und inkonsistenten Dateiablage. Dieser Zustand führt zu erheblichen Herausforderungen in Bezug auf Datenkonsistenz, Mehrbenutzerzugriff und Informationssicherheit. Da die Daten oft redundant in verschiedenen Tabellenkalkulationsdokumenten vorliegen, ist eine fehlerfreie Aktualisierung kaum möglich.

Ziel dieses Praxisbelegs ist es, eine zentrale, datenbankgestützte Lösung zu konzipieren und zu implementieren, die einen plattformunabhängigen Zugriff ermöglicht und gleichzeitig höchste Sicherheitsstandards erfüllt.

\chapter{Analyse und Konzeption}
\section{Ist-Analyse und Datenbereinigung}
Die initiale Analyse der vorhandenen Daten auf dem NAS offenbarte eine hohe Fehlerquote. Viele Datensätze waren unvollständig oder lagen in veralteten Versionen vor. Im ersten Schritt der Projektarbeit wurden diese Daten manuell gesichtet, bereinigt und in ein einheitliches Format überführt. Dabei wurden Dubletten entfernt und fehlende Pflichtfelder (z. B. E-Mail-Adressen und Steuernummern) ergänzt. Diese bereinigte Datenbasis bildet das Fundament für die Migration in das neue relationale Datenbanksystem.

\section{Soll-Konzept und Architektur}
Das neue System basiert auf einer klassischen Client-Server-Architektur. Um Flexibilität und Skalierbarkeit zu gewährleisten, wird die gesamte Server-Infrastruktur mithilfe von Docker-Containern virtualisiert. Das Herzstück bildet die PostgreSQL-Datenbank, die über eine REST-API mit den Clients kommuniziert.

\bild[0.6]{architektur_diagramm}{Systemarchitektur der zentralen Kundendatenverwaltung}{fig:architektur}

Wie in \literef{fig:architektur} zu sehen ist, fungiert ein Nginx-Server als \striche{Reverse Proxy}, der die verschlüsselten Anfragen an die internen Dienste weiterleitet. Dieser Aufbau ermöglicht eine saubere Trennung der Verantwortlichkeiten und eine einfache Wartbarkeit.

\chapter{Technische Umsetzung}
\section{Backend und Infrastruktur}
\subsection{Datenhaltung und PostgreSQL}
Als Datenbanksystem wurde PostgreSQL gewählt, da es robuste Funktionen für die Datenintegrität bietet. Ein entscheidendes Merkmal für dieses Projekt ist die \striche{Row Level Security} (RLS). Damit wird sichergestellt, dass Nutzer (z. B. Praktikanten vs. Mitarbeiter) nur Zugriff auf die Zeilen erhalten, für die sie explizit berechtigt sind\onlinezitat{postgresql_docs}. Dies erhöht die Datensicherheit massiv, da die Zugriffskontrolle direkt auf Datenbankebene und nicht nur in der Applikationslogik stattfindet.

\bild[0.5]{ERD}{Ein Ausschnitt des ERD der PostgreSQL Datenbank}{fig:ERD}

\subsection{API-Entwicklung mit FastAPI}
Die Kommunikation zwischen Frontend und Datenbank erfolgt über eine REST-Schnittstelle, die mit Python und dem Framework FastAPI realisiert wurde\onlinezitat{fastapi_docs}. FastAPI zeichnet sich durch seine hohe Performance und die automatische Generierung von API-Dokumentationen aus.

Für den produktiven Betrieb wird FastAPI hinter einem Gunicorn-Server mit vier Uvicorn-Workern betrieben, um parallele Anfragen effizient verarbeiten zu können.

\section{Frontend-Entwicklung mit Flutter}
Um den Mitarbeitern den Zugriff auf verschiedenen Endgeräten (Windows, Linux, Android, iOS) zu ermöglichen, wurde das Frontend mit dem Framework Flutter entwickelt\onlinezitat{flutter_dev}. Flutter erlaubt es, eine einzige Codebasis für alle Plattformen zu nutzen, was den Wartungsaufwand erheblich reduziert.

\bild[1]{flutter_ui}{Screenshot von dem UI der Verwaltungs-App}{fig:flutter}

Die App kommuniziert gesichert via HTTPS mit dem Backend und nutzt \striche{JSON Web Tokens} (JWT) zur Authentifizierung der Benutzer.

\section{Sicherheitskonzept}
Die Sicherheit der Kundendaten steht an oberster Stelle. Der Server (Ubuntu 22.04) ist durch eine LUKS-Verschlüsselung der Festplatten sowie ein RAID-1-System vor physikalischem Datenverlust und Diebstahl geschützt. Der Zugriff von außerhalb des Firmennetzwerkes ist ausschließlich über einen gesicherten VPN-Tunnel möglich. Innerhalb der Anwendung sorgt die JWT-basierte Authentifizierung in Kombination mit der PostgreSQL-RLS für eine feingranulare Berechtigungssteuerung.

\chapter{Zusammenfassung und Ausblick}
Durch die Implementierung der zentralen Kundendatenverwaltung wurde die Fehlerquote bei der Datenpflege drastisch gesenkt. Die gewählte Architektur aus FastAPI und Flutter bietet eine zukunftssichere Basis für weitere Unternehmenserweiterungen. Im nächsten Schritt ist die Anbindung eines Dokumentenmanagementsystems geplant, um auch Verträge direkt den Kundendaten zuzuordnen.